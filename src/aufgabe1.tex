\documentclass[12pt,german,a4]{article}

\usepackage{ngerman}
\usepackage[utf8]{inputenc}
\usepackage[T1]{fontenc}
\usepackage{amsthm}
\usepackage{amsfonts}
\usepackage{amssymb}
\usepackage{amsmath}
\setlength{\parindent}{0pt}

\begin{document}

{\bf Satz}: Abzählbare Vereinigungen abzählbarer Mengen sind wieder abzählbar.\\

{\bf Beweis}:

Durch Umsortieren ist leicht zu erkennen, dass gilt:\\

(1): $\mathbb{N} \times \mathbb{N} = \sum_{n \in \mathbb{N}}{(\mathbb{N} \times \{n\})}$ \\

Sei nun $B_{i}~ (i \in \mathbb{N})$ eine Familie abzählbarer Mengen.\\

(2): $\forall i \in \mathbb{N}: B_{i}$ abzählbar $\implies \exists f_{i} : B_{i} \rightarrow \mathbb{N}: f_{i}$ injektiv\\

(3): $\bigcup\{B_{i} : i \in \mathbb{N}\} = \sum\{B_{i} - \bigcup\{B_{j} : j < i\} : i \in \mathbb{N}\}$\\

Da dies jeweils Teilmengen sind, gilt:\\

(4): $(2) \wedge (3) \implies \forall i \in \mathbb{N}: \exists g_{i} :( B_{i} - \sum\{B_{j}: j < i\}) \rightarrow (\mathbb{N} \cup \{i\}): g_{i}$ injektiv\\

Somit können wir die disjunkte Vereinigung komponentenweise nach $\mathbb{N} \times \mathbb{N}$ abbilden:\\

(5): $(1) \wedge (4) \implies \exists h:  \sum\{B_{i} - \bigcup\{B_{j} : j < i\} : i \in \mathbb{N}\} \rightarrow \mathbb{N} \times \mathbb{N}: h$ injektiv\\

(6): $(3) \wedge (5) \implies \exists h: \bigcup\{B_{i} : i \in \mathbb{N}\} \rightarrow \mathbb{N} \times \mathbb{N}: h$ injektiv\\

Jetzt verwenden wir noch die injektive Abbildung $\varphi : \mathbb{N} \times \mathbb{N} \rightarrow \mathbb{N}$ aus Aufgabenteil 2:\\

(7): $(6) \implies \exists h;\varphi : \bigcup\{B_{i} : i \in \mathbb{N}\} \rightarrow \mathbb{N}: h;\varphi$ injektiv\\

(8): $(7) \implies \bigcup\{B_{i} : i \in \mathbb{N}\}$ ist abzählbar. $\qed$

\end{document}